\documentclass[]{article}
\usepackage{lmodern}
\usepackage{amssymb,amsmath}
\usepackage{ifxetex,ifluatex}
\usepackage{fixltx2e} % provides \textsubscript
\ifnum 0\ifxetex 1\fi\ifluatex 1\fi=0 % if pdftex
  \usepackage[T1]{fontenc}
  \usepackage[utf8]{inputenc}
\else % if luatex or xelatex
  \ifxetex
    \usepackage{mathspec}
  \else
    \usepackage{fontspec}
  \fi
  \defaultfontfeatures{Ligatures=TeX,Scale=MatchLowercase}
\fi
% use upquote if available, for straight quotes in verbatim environments
\IfFileExists{upquote.sty}{\usepackage{upquote}}{}
% use microtype if available
\IfFileExists{microtype.sty}{%
\usepackage{microtype}
\UseMicrotypeSet[protrusion]{basicmath} % disable protrusion for tt fonts
}{}
\usepackage[margin=1in]{geometry}
\usepackage{hyperref}
\hypersetup{unicode=true,
            pdftitle={STAT 702 - Final Project},
            pdfborder={0 0 0},
            breaklinks=true}
\urlstyle{same}  % don't use monospace font for urls
\usepackage{color}
\usepackage{fancyvrb}
\newcommand{\VerbBar}{|}
\newcommand{\VERB}{\Verb[commandchars=\\\{\}]}
\DefineVerbatimEnvironment{Highlighting}{Verbatim}{commandchars=\\\{\}}
% Add ',fontsize=\small' for more characters per line
\usepackage{framed}
\definecolor{shadecolor}{RGB}{248,248,248}
\newenvironment{Shaded}{\begin{snugshade}}{\end{snugshade}}
\newcommand{\KeywordTok}[1]{\textcolor[rgb]{0.13,0.29,0.53}{\textbf{#1}}}
\newcommand{\DataTypeTok}[1]{\textcolor[rgb]{0.13,0.29,0.53}{#1}}
\newcommand{\DecValTok}[1]{\textcolor[rgb]{0.00,0.00,0.81}{#1}}
\newcommand{\BaseNTok}[1]{\textcolor[rgb]{0.00,0.00,0.81}{#1}}
\newcommand{\FloatTok}[1]{\textcolor[rgb]{0.00,0.00,0.81}{#1}}
\newcommand{\ConstantTok}[1]{\textcolor[rgb]{0.00,0.00,0.00}{#1}}
\newcommand{\CharTok}[1]{\textcolor[rgb]{0.31,0.60,0.02}{#1}}
\newcommand{\SpecialCharTok}[1]{\textcolor[rgb]{0.00,0.00,0.00}{#1}}
\newcommand{\StringTok}[1]{\textcolor[rgb]{0.31,0.60,0.02}{#1}}
\newcommand{\VerbatimStringTok}[1]{\textcolor[rgb]{0.31,0.60,0.02}{#1}}
\newcommand{\SpecialStringTok}[1]{\textcolor[rgb]{0.31,0.60,0.02}{#1}}
\newcommand{\ImportTok}[1]{#1}
\newcommand{\CommentTok}[1]{\textcolor[rgb]{0.56,0.35,0.01}{\textit{#1}}}
\newcommand{\DocumentationTok}[1]{\textcolor[rgb]{0.56,0.35,0.01}{\textbf{\textit{#1}}}}
\newcommand{\AnnotationTok}[1]{\textcolor[rgb]{0.56,0.35,0.01}{\textbf{\textit{#1}}}}
\newcommand{\CommentVarTok}[1]{\textcolor[rgb]{0.56,0.35,0.01}{\textbf{\textit{#1}}}}
\newcommand{\OtherTok}[1]{\textcolor[rgb]{0.56,0.35,0.01}{#1}}
\newcommand{\FunctionTok}[1]{\textcolor[rgb]{0.00,0.00,0.00}{#1}}
\newcommand{\VariableTok}[1]{\textcolor[rgb]{0.00,0.00,0.00}{#1}}
\newcommand{\ControlFlowTok}[1]{\textcolor[rgb]{0.13,0.29,0.53}{\textbf{#1}}}
\newcommand{\OperatorTok}[1]{\textcolor[rgb]{0.81,0.36,0.00}{\textbf{#1}}}
\newcommand{\BuiltInTok}[1]{#1}
\newcommand{\ExtensionTok}[1]{#1}
\newcommand{\PreprocessorTok}[1]{\textcolor[rgb]{0.56,0.35,0.01}{\textit{#1}}}
\newcommand{\AttributeTok}[1]{\textcolor[rgb]{0.77,0.63,0.00}{#1}}
\newcommand{\RegionMarkerTok}[1]{#1}
\newcommand{\InformationTok}[1]{\textcolor[rgb]{0.56,0.35,0.01}{\textbf{\textit{#1}}}}
\newcommand{\WarningTok}[1]{\textcolor[rgb]{0.56,0.35,0.01}{\textbf{\textit{#1}}}}
\newcommand{\AlertTok}[1]{\textcolor[rgb]{0.94,0.16,0.16}{#1}}
\newcommand{\ErrorTok}[1]{\textcolor[rgb]{0.64,0.00,0.00}{\textbf{#1}}}
\newcommand{\NormalTok}[1]{#1}
\usepackage{longtable,booktabs}
\usepackage{graphicx,grffile}
\makeatletter
\def\maxwidth{\ifdim\Gin@nat@width>\linewidth\linewidth\else\Gin@nat@width\fi}
\def\maxheight{\ifdim\Gin@nat@height>\textheight\textheight\else\Gin@nat@height\fi}
\makeatother
% Scale images if necessary, so that they will not overflow the page
% margins by default, and it is still possible to overwrite the defaults
% using explicit options in \includegraphics[width, height, ...]{}
\setkeys{Gin}{width=\maxwidth,height=\maxheight,keepaspectratio}
\IfFileExists{parskip.sty}{%
\usepackage{parskip}
}{% else
\setlength{\parindent}{0pt}
\setlength{\parskip}{6pt plus 2pt minus 1pt}
}
\setlength{\emergencystretch}{3em}  % prevent overfull lines
\providecommand{\tightlist}{%
  \setlength{\itemsep}{0pt}\setlength{\parskip}{0pt}}
\setcounter{secnumdepth}{0}
% Redefines (sub)paragraphs to behave more like sections
\ifx\paragraph\undefined\else
\let\oldparagraph\paragraph
\renewcommand{\paragraph}[1]{\oldparagraph{#1}\mbox{}}
\fi
\ifx\subparagraph\undefined\else
\let\oldsubparagraph\subparagraph
\renewcommand{\subparagraph}[1]{\oldsubparagraph{#1}\mbox{}}
\fi

%%% Use protect on footnotes to avoid problems with footnotes in titles
\let\rmarkdownfootnote\footnote%
\def\footnote{\protect\rmarkdownfootnote}

%%% Change title format to be more compact
\usepackage{titling}

% Create subtitle command for use in maketitle
\newcommand{\subtitle}[1]{
  \posttitle{
    \begin{center}\large#1\end{center}
    }
}

\setlength{\droptitle}{-2em}
  \title{STAT 702 - Final Project}
  \pretitle{\vspace{\droptitle}\centering\huge}
  \posttitle{\par}
  \author{}
  \preauthor{}\postauthor{}
  \predate{\centering\large\emph}
  \postdate{\par}
  \date{Tuesday April 17, 2018}

\usepackage{booktabs}
\usepackage{float}
\let\origfigure\figure
\let\endorigfigure\endfigure
\renewenvironment{figure}[1][2] {
    \expandafter\origfigure\expandafter[H]
} {
    \endorigfigure
}
\usepackage{lscape}
\newcommand{\blandscape}{\begin{landscape}}
\newcommand{\elandscape}{\end{landscape}}
\usepackage{fullpage}
\usepackage{pdflscape}
\newcommand{\bpdflandscape}{\begin{landscape}}
\newcommand{\epdflandscape}{\end{landscape}}

\begin{document}
\maketitle

{
\setcounter{tocdepth}{2}
\tableofcontents
}
\allowdisplaybreaks
\newpage

\section{Abstract}\label{abstract}

This paper provides an introduction on the background of this data set
and indicates the goals of our modelling efforts in relation to the
stated problem. Following this section, we provide a methodology of
several various models after initial data set and variable consideration
with the appropriate changes being made. We present the corresponding
test error rates using the validation set approach and \emph{K}-fold
cross validation of these models along with other model analyzing
metrics and provide a suggestion on the model which provides the best
predictive ability for the \emph{known} keystroke dynamic data set.
Following this suggestion, the \emph{unknown} keystroke dynamic data set
was used to bolster the findings of the suggested model along with a
summary of conclusions of our findings.

\section{Introduction}\label{introduction}

The use of keystroke dynamics collects information on the typing
patterns of a person(s) in a detailed timing manner, which can and has
been used to develop additional security measures for online and/or
passcode sites (i.e.~banks, etc.) {[}1{]} and/or identify different
individual(s). In this current paper, two data sets were provided with
keystroke dynamic information in which subjects entered the same
passcode, ``.tie5Roan1'' to gain access to a protected system. One data
set contained known subjects who entered keystroke information and the
other with unknown subjects entering the same information (hereafter
referred to as \emph{known} and \emph{unknown}, respectively). The both
data sets listed data in a hierarchal format to identify each time the
passcode was entered (i.e., an individual observation) with the
\emph{known} data set consisting of the known subject (\emph{subject}),
a designated session (\emph{sessionIndex}), and the number of times
(i.e., replicates) each subject entered the passcode for a given session
(\emph{rep}). The \emph{unknown} data set contained designated sessions
(\emph{sessionID}) along with number of times the passcode was entered
for each session (\emph{rep}) but no information on who entered it. Both
data sets contained an additional 31 variables consisting of the
keystroke dynamic information for the listed passcode; a detailed
description for each of these variables is listed in Appendi
\ref{appendix:known.vars}.

The overall goals of this project and resulting paper were to first
develop several different models (i.e., classifiers) following
exploratory analysis and variable selection using the \emph{known} data
set in order to find a model with the best predictive ability for
determining subjects from unknown keystroke dynamics information. Our
second goal was to analyzing performance metrics of these developed
models to determine the model with the best predictive ability (i.e.,
lowest test error rate) from the \emph{known} data set. Our final goal
included the use of the \emph{unknown} data set within the final
selected model to try to improve the model's accuracy. The steps used to
address these goals are listed in the following sections.

\section{Methods and Results}\label{methods-and-results}

\subsection{Exploratory Data Analysis and Variable
Selection}\label{exploratory-data-analysis-and-variable-selection}

Prior to developing any models, exploratory graphs were developed in
order to assist in variable selection and model development. Also, to
make plotting and visual analysis easier, the original names/values of
the \texttt{subject} variable are replaced, this could be seen in
Appendix \ref{appendix:shortened.subject.map}. The first set of graphs
developed consisted of a several series of ridge plots depicting the
distributions of time (milliseconds) for each subject: individual
character hold in the passcode (Hold), key transition for down-down
(DD), and key transition for up-down (UD). Since this plotting looked at
each individual, a large amount of plots were developed; therefore, only
several sample ones (Figure \ref{figure:heatmap}) are presented here,
while the remaining plots are listed in Appendix 2. While these plots
were informative, the overall number of subjects and variables made
these plots quite cumbersome to analyze; therefore, an additional heat
map was created depicting the same information but allowing us to look
at each subject across all three differing keystroke information sets as
listed above (Figure 2). Along with exploring the \emph{known} data set
at the subject level, ridge, violin, and box plots were made depicting
the total distributions of all combined subjects along with outliers
with the same keystroke information sets (Figure
\ref{figure:overall-distribution-keyhold},
\ref{figure:overall-distribution-DD},
\ref{figure:overall-distribution-UD}).

\begin{Shaded}
\begin{Highlighting}[]
\KeywordTok{plotHeatMap}\NormalTok{(known.Long.PerSubject)}
\end{Highlighting}
\end{Shaded}

\begin{figure}
\centering
\includegraphics{Final_files/figure-latex/unnamed-chunk-4-1.pdf}
\caption{Heat map of every subject for three different sets of keystroke
data sets (i.e., Key Up-Down, Key Down-Down, and Key hold).
\label{figure:heatmap}}
\end{figure}

\begin{Shaded}
\begin{Highlighting}[]
\KeywordTok{plotKeyActionSummary}\NormalTok{(known.Long.PerSubject }\OperatorTok\StringTok{ }\KeywordTok{filter}\NormalTok{(Key.Action }\OperatorTok{==}\StringTok{ }\NormalTok{KEY_ACTION_HOLD), }\StringTok{"Key Action - Hold"}\NormalTok{)}
\end{Highlighting}
\end{Shaded}

\begin{figure}
\centering
\includegraphics{Final_files/figure-latex/unnamed-chunk-5-1.pdf}
\caption{Ridgle plot, violin plot, and barplot of depicting overall
distribution of Up-Down keystroke information.
\label{figure:overall-distribution-keyhold}}
\end{figure}

\begin{Shaded}
\begin{Highlighting}[]
\KeywordTok{plotKeyActionSummary}\NormalTok{(known.Long.PerSubject }\OperatorTok\StringTok{ }\KeywordTok{filter}\NormalTok{(Key.Action }\OperatorTok{==}\StringTok{ }\NormalTok{KEY_ACTION_DOWN_DOWN), }\StringTok{"Key Action - Down-Down"}\NormalTok{)}
\end{Highlighting}
\end{Shaded}

\begin{figure}
\centering
\includegraphics{Final_files/figure-latex/unnamed-chunk-6-1.pdf}
\caption{Ridgle plot, violin plot, and barplot of depicting overall
distribution of Down-Down keystroke information.
\label{figure:overall-distribution-DD}}
\end{figure}

\begin{Shaded}
\begin{Highlighting}[]
\KeywordTok{plotKeyActionSummary}\NormalTok{(known.Long.PerSubject }\OperatorTok\StringTok{ }\KeywordTok{filter}\NormalTok{(Key.Action }\OperatorTok{==}\StringTok{ }\NormalTok{KEY_ACTION_UP_DOWN), }\StringTok{"Key Action - Up-Down"}\NormalTok{)}
\end{Highlighting}
\end{Shaded}

\begin{figure}
\centering
\includegraphics{Final_files/figure-latex/unnamed-chunk-7-1.pdf}
\caption{Ridgle plot, violin plot, and barplot of depicting overall
distribution of Up-Down keystroke information.
\label{figure:overall-distribution-UD}}
\end{figure}

Analysis of the ridge plots indicated a large amount of similar
distributions occurring over a wide range of subjects and keystroke
information. Likewise, the heat maps depicted differences occurring
between different subjects within each keystroke set. However, this
graph also pointed to the possibility of collinearity occurring do to
the similar trends seen between the key transition for DD and UD, which
is addressed later in this section. For the overall distribution plots,
very similar patterns were seen in the UD and DD information, while the
Hold information contained much wider distributions. Also, these same
plots, especially the boxplots, indicated that the data set maintains a
high amount of potential outliers. While several different approaches
could be taken to address this issue such as removing observations for
justifiable reasons, due to the limited amount of background knowledge
on this data set, we felt it would be unwise to remove such observations
without knowing the true nature behind their variation. Therefore, we
decided to use a conservative approach by including all the data points
within our models, knowing reduced model predictability may occur, in
order to maintain a high level of statistical integrity. Such data
points could be reviewed and removed at a later date after more
background information is garnered on those observations in question.

As mentioned before, several of the plots suggested the possibility of
collinearity between variables; therefore, both scatterplot and
correlation matrices were created to test if this occurred. These
matrices indicated collinearity occurred between the keystroke
information between UD and DD keystroke times for all typed components
of the passcode with a pearson correlation coefficient greater than 0.9
and clearly depicted linear relationship seen in Figure
\ref{figure:collinearity} (due to the overall size of the scatterplot
matrix from the 51 subjects, only plots of the variables with
collinearity are shown here).

\begin{figure}
\centering
\includegraphics{Final_files/figure-latex/unnamed-chunk-8-1.pdf}
\caption{Scatterplot matrix between Down-Down and Up-Down keystroke
infomation for all variables showing collinearity.
\label{figure:collinearity}}
\end{figure}

In order to reduce potential problem with model performance and
overfitting from collinearity, we decided to remove either the UD or the
DD variables. To help choose which variable to remove, we created two
models with the first containing all variables except UD variables and
the second with all variables except DD variables. A quadratic
discriminant analysis (QDA) model was fit to determine which set of
predictors made a better model based on calculated test errors. From the
validation set approach (VSA), the test error was lower for the model
that used the UD variable; therefore, it was decided that for all future
models, all DD variables would be removed.

Another consideration taken into account was the \emph{sessionIndex}
variable in the \emph{known} data set. This variable had only two
different values; 7 or 8 with some subjects having a 7, some an 8, and
others having both. Such differences led us to question if potential
differences between the 7's and 8's for each covariate occurred. Because
the distribution of the covariates for each of the groups is not known,
31 non-parametric tests were run to determine significance for each
variable. Additionally, we decided to only use those observations that
had both a 7 and an 8 which were analyzed with a Wilcoxon rank-sign
test. The p-values for each of the variables can be found in the
appendix. To adjust for multiple testing, we used the Bonferroni and
Benjamini-Hochberg correction factors. The Bonferroni correction factor
was extremely conservative, while the Benjamini-Hochberg \emph{P}-value
indicated a significant difference (i.e., rejected the null hypothesis)
between means of the first 5 variables rejected the null hypothesis
assuming an \(\alpha = 0.05\). Therefore, we proceeded with caution, but
choose use both the 7 and 8 together in the training data set. The
\emph{known} data set was split with a 70/30 ratio between training/test
data sets; barplots of the training and test data sets were created and
ensured that all subjects were represented in both data sets (Figure
\ref{figure:train.test})

\begin{figure}
\centering
\includegraphics{Final_files/figure-latex/unnamed-chunk-9-1.pdf}
\caption{Barplots of number of subject observations included in the
known, training, and test, datasets. \label{figure:train.test}}
\end{figure}

\subsection{Model Methodology}\label{model-methodology}

Due to the nature of the categorical response variable, regular binary
modeling techniques as they are structured in our book would be ill
suited for modelling. However, we decided to use a one-vs-all approach
in order to create a ``quasi-binary'' outcome for each of the 51
subjects for several of the models mentioned below. The process of this
technique occurred as follows: 1) one class was chosen, 2) a new
dependent variable is created with a value of 1 if the observation is in
the class or 0 if otherwise and 3) a given model is fit using the
original covariates on the created binary dependent variable {[}2{]}.
This process is performed iteratively for all of the different subject
values resulting in 51 different models each with a different set of
estimated parameters that are estimated using the typical logistic
regression maximum likelihood estimation.

\subsection{Parametric Modelling}\label{parametric-modelling}

In order to implement this algorithm within a parametric model, we
created a function that fits a logistic regression model which iterated
through different cutoff points in order to determine the best cutoff
for that particular logistic regression model. After running this
function on the \emph{known} data set, warnings occurred indicating the
algorithm did not converge along with fitted probabilities of 0 or 1.
Analysis of the model output found that many of the parameter estimates
did not make sense including standard error estimates that were
extremely large for many of the parameter estimates. These results
suggested that the model created a situation where full separation or
quasi-separation occurred, likely due to the one-vs-all approach. This
situation resulted due to limited crossover between the covariates for
the two response outcomes which caused the results to become unreliable.
(A 2-d visualization of this is given below.)

\begin{figure}
\centering
\includegraphics{Final_files/figure-latex/unnamed-chunk-10-1.pdf}
\caption{Example of 2-d separation \label{figure:separation}}
\end{figure}

Two additional models considered in development using the known data set
were a linear discriminant analysis (LDA), and quadratic discriminant
analysis (QDA) which were created, performed, and tested for accuracy
using the VSA. Unlike, the logistic model mentioned above, this model
produced reliable results which resulted in a VSA test error of 27.77\%.
Due to this relatively large error rate (given the current data set and
goals) at hand along with unreliable results from the logistic
regression model, it appeared that the usage of non-parametric
techniques would be more suitable and are discussed in the next section.

\subsection{Non- and Semi-Parametric}\label{non--and-semi-parametric}

\subsubsection{K-nearest neighbors}\label{k-nearest-neighbors}

Another modeling method analyzed on the \emph{known} data set was
K-nearest neighbors (KNN) classification. We decided to used 10-fold
cross validation to obtain 10 separate test error rates to determine a
value for \emph{K}. These test error rates were then averaged to obtain
a single error rate for a given value of \emph{K}, which were
incremented through values of \emph{K} from 1 to 100. After doing this,
we get the following plot referenced in the Figure \ref{figure:KNN}.
This plot showed that a generally increasing trend in the error rate
occurred as the value of \emph{K} increased. The value of \(K=1\)
resulted in the lowest error rate. This may be due to the correlation
within each of the repetitions within each session. The error rate
corresponding to a \(K=1\) was 23.29\%, indicating poor model
performance.

\begin{figure}
\centering
\includegraphics{Final_files/figure-latex/unnamed-chunk-12-1.pdf}
\caption{Mean 10-fold CV error rates for different K in KNN classifier.
\label{figure:KNN}}
\end{figure}

\subsubsection{Classification Tree}\label{classification-tree}

Using the full formular, a multi-class classification tree is created
using the \emph{rpart} library (\emph{We tried tree, but it doesn't seem
to be able to handle more that 32 classes}). Before fitting the tree,
the \emph{tune} function is ussed to tune the \emph{cp}, \emph{minbucket
(min \# of observation in terminal node)} and \emph{maxdepth (max depth
of tree)} hyperparameters, achieved through 10-fold cross validation.
Using the optimal hyperparameters, a multi-class classification tree is
created. Using 6-fold cross validation on all data in \emph{known.csv},
an average test error of \(35.75\%\) is obtained. This is quite poor.

\subsubsection{Random Forest}\label{random-forest}

Given the poor performance of the \emph{rpart} model, we'll explore some
other forms of classification tree models. Specifically looking into
Bagging, and RandomForest. Given the computation intensiveness of these
algoriuthim, we decided to limit the space where we try to optimize our
hyerparameters, in this case \emph{ntree} and \emph{mtry}. In Figure
\ref{fig-rf-mtry-errors}, we see the errors obtained from differtn
combinatio of the hyperparameters. On each combination, an
\emph{Out-of-Bag (OOB)} and \emph{test} error is obtained, using 6-fold
cross validation, and VSA.

Although according to Leo Breiman, \emph{\ldots{}In random forests,
there is no need for cross-validation or a separate test set to get an
unbiased estimate of the test set error\ldots{}} Following this tredn of
tought, it could be seen in \ref{fig-rf-mtry-errors}, that for the
errors obtained without CV, when compared to test error, the OOB is
noticably bigger across the board. Hinting that using the OOB error as a
form of model goodness, a better model could be achieved. With this, the
mtry of 4 is used for fitting a random forest of 1000 trees.

Fiting this new model with the traiing dataset, a test error of
\(3.56\%\) is obtained. This is quite an improvement when compared to

Using the full formular, a multi-class classification tree is created
using the \emph{rpart} library (\emph{We tried tree, but it doesn't seem
to be able to handle more that 32 classes}). Before fitting the tree,
the \emph{tune} function is ussed to tune the \emph{cp}, \emph{minbucket
(min \# of observation in terminal node)} and \emph{maxdepth (max depth
of tree)} hyperparameters, achieved through 10-fold cross validation.
Using the optimal hyperparameters, a multi-class classiifcation tree is
created. Using 6 fold cross validation on all data in \emph{known.csv},
an average test error of \(35.75\%\) is obtained. This is quite poor.

\begin{figure}
\centering
\includegraphics{Final_files/figure-latex/unnamed-chunk-16-1.pdf}
\caption{Error rates accross mtry values\label{fig-rf-mtry-errors}}
\end{figure}

\subsection{Support Vector Machine}\label{support-vector-machine}

\subsubsection{Random Forest}\label{random-forest-1}

To fit a Support Vector Machine model, the \emph{e1071} library is used.
Unlike the One-vs-All approach we've used prior, this library uses
One-vs-One to handle multiclass. For the SVM, we explore 3 kernel basis
function, \emph{linear}, \emph{polynomial}, and \emph{radial}. The tune
function is used to obtain the optimal hyperparameters:

\begin{itemize}
\tightlist
\item
  linear - cost(c)
\item
  polynomial - cost(c), degree
\item
  radial - cost(c), gamma
\end{itemize}

Using the optimal hyperparameters, we fit 3 SVM models. Table
\ref{table:error.svm.models} shows the performance of all 3 models,
where the linear and radial kernels perform best.

\begin{table}[!htb]
  \caption{Error rates for all 3 SVM models, with various kernel function}
  \label{table:error.svm.models}
  \begin{tabular}{ l | c | r }
    \text{linear} & 11.44\%\\
    \text{polynomial} & 12.95\%\\
    \text{radial} & 10.51\% \\
  \end{tabular}
\end{table}

\subsection{Final Model with Unknown Data
Set}\label{final-model-with-unknown-data-set}

(Add paragraph on why we selected random forest as best along with
supporting evidence - table of test error rates from all models)

In an attempt to improve the predictive ability of our selected model,
the \emph{unknown} data set was analyzed and used as part of the model
building process. To do this, initial attempts at clustering with both
k-means and Mclust clustering algorithms, indicated that unclear cuts
between groups were not formed. Therefore, a level of subjectivity would
need to be incorporated in order to decide what cluster belonged to what
subject. Instead of this procedure, we decided to infer predictions
based on a random forest. A tuned random forest was first trained on the
\emph{known} data set then used to predict the \emph{subjects} in the
\emph{unknown} data set. Each row associated with a unique
\emph{sessionID} was predicted, however it was not the same prediction
for each row because of the differences in the covariates. The
\emph{subject} with the highest number of predictions within each
\emph{sessionID} was then assumed to represent the unkown subject for
that \emph{sessionID}. After the subjects were predicted and assigned in
the \emph{unknown} data set, this data was appended to the \emph{known}
training data set and validated on the \emph{known} test data set. (Need
to add results found from this procedure)

\section{Conclusions}\label{conclusions}

\section{Report References}\label{report-references}

{[}1{]} Parimarjan, Negi. Adversarial machine learning against keystroke
dynamics. PDF on \url{http://www.Semanticsscholar.org}. Accessed: April
23, 2018.

{[}2{]} Rifkin, Ryan. Multiclass classification. PDF on
\url{http://www.mit.edu/~9.520/spring09/Classes/} multiclass.pdf.
Accessed: April 26, 2018.

{[}3{]} Breiman, Leo, and Adele Cutler. Random Forests. HTML on
\url{https://www.stat.berkeley.edu/~breiman/RandomForests/cc_home.htm\#ooberr}.
Accessed: April 26, 2018.

\section{Code References}\label{code-references}

\begin{enumerate}
\def\labelenumi{\arabic{enumi})}
\item
  Source:
  \url{https://tex.stackexchange.com/questions/2832/how-can-i-have-two-tables-side-by-side}.
  Accessed April 26, 2018.
\item
  Source:
  \url{https://stackoverflow.com/questions/38036680/align-multiple-tables-side-by-side}.
  Accessed April 26, 2018.
\item
  Source:
  \url{https://tex.stackexchange.com/questions/8652/what-does-t-and-ht-mean}.
  Accessed April 26, 2018.
\item
  Source:
  \url{https://stackoverflow.com/questions/45409750/get-rid-of-addlinespace-in-kable}.
  Accessed April 26, 2018.
\item
  Source:
  \url{https://www.r-bloggers.com/ggplot2-easy-way-to-mix-multiple-graphs-on-the-same-page/}.
  Accessed April 27, 2018
\end{enumerate}

\section{Appendices}\label{appendices}

\subsection{\texorpdfstring{\label{appendix:known.vars}}{}}\label{section}

\begin{longtable}[]{@{}ll@{}}
\caption{Dictionary for the \emph{known.csv}}\tabularnewline
\toprule
Variable & Description\tabularnewline
\midrule
\endfirsthead
\toprule
Variable & Description\tabularnewline
\midrule
\endhead
H.period & The amount of time that the ``.'' is held
down.\tabularnewline
DD.period.t & The time between pressing down the ``.'' key to the time
to press down the ``t'' key.\tabularnewline
UD.period.t & The time between the ``.'' key coming up to the time to
press down the ``t'' key.\tabularnewline
H.t & The amount of time that the ``t'' is held down.\tabularnewline
DD.t.i & The time between pressing down the ``t'' key to the time to
press down the ``i'' key.\tabularnewline
UD.t.i & The time between the ``t'' key coming up to the time to press
down the ``i'' key.\tabularnewline
H.i & The amount of time that the ``i'' is held down.\tabularnewline
DD.i.e & The time between pressing down the ``i'' key to the time to
press down the ``e'' key.\tabularnewline
UD.i.e & The time between the ``i'' key coming up to the time to press
down the ``e'' key.\tabularnewline
H.e & The amount of time that the ``e'' is held down.\tabularnewline
DD.e.five & The time between pressing down the ``e'' key to the time to
press down the ``5'' key.\tabularnewline
UD.e.five & The time between the ``e'' key coming up to the time to
press down the ``5'' key.\tabularnewline
H.five & The amount of time that the ``5'' is held down.\tabularnewline
DD.five.Shift.r & The time between pressing down the ``5'' key to the
time to press down the ``shift+r'' key combination.\tabularnewline
UD.five.Shift.r & The time between the ``5'' key coming up to the time
to press down the ``shift+r'' key combination.\tabularnewline
H.Shift.r & The amount of time that the ``shift+r'' key combination is
held down.\tabularnewline
DD.Shift.r.o & The time between pressing down the ``shift+r'' key
combination to the time to press down the ``o'' key.\tabularnewline
UD.Shift.r.o & The time between the ``shift+r'' key combination coming
up to the time to press down the ``o'' key.\tabularnewline
H.o & The amount of time that the ``o'' is held down.\tabularnewline
DD.o.a & The time between pressing down the ``o'' key to the time to
press down the ``a'' key.\tabularnewline
UD.o.a & The time between the ``o'' key coming up to the time to press
down the ``a'' key.\tabularnewline
H.a & The amount of time that the ``a'' is held down.\tabularnewline
DD.a.n & The time between pressing down the ``a'' key to the time to
press down the ``n'' key.\tabularnewline
UD.a.n & The time between the ``a'' key coming up to the time to press
down the ``n'' key.\tabularnewline
H.n & The amount of time that the ``n'' is held down.\tabularnewline
DD.n.l & The time between pressing down the ``n'' key to the time to
press down the\tabularnewline
UD.n.l & The time between the ``n'' key coming up to the time to press
down the ``l'' key.\tabularnewline
H.l & The amount of time that the ``l'' is held down.\tabularnewline
DD.l.Return & The time between pressing down the ``l'' key to the time
to press down the ``return'' key.\tabularnewline
UD.l.Return & The time between the ``l'' key coming up to the time to
press down the ``return'' key.\tabularnewline
H.Return & The amount of time that the ``return'' is held
down.\tabularnewline
session & A session is a block of time where an individual has had
access to the system over a continuous block of time.\tabularnewline
rep & The individual passcode entries within a session are referred to
as a repetition-within-session of the passcode entry.\tabularnewline
subject & IDs for all 51 unique individuals.\tabularnewline
\bottomrule
\end{longtable}

\subsection{\texorpdfstring{\label{appendix:shortened.subject.map}}{}}\label{section-1}

\begin{table}[!htb]
  \caption{Mapping for the renamed \textit{subject} variable}
  \label{table:shortened.subject.map}
  \begin{minipage}{.33\linewidth}
    \centering
      
\begin{tabular}{ll}
\toprule
Old Subject & New Subject\\
\midrule
s002 & a\\
s003 & b\\
s004 & c\\
s005 & d\\
s007 & e\\
s008 & f\\
s010 & g\\
s011 & h\\
s012 & i\\
s013 & j\\
s015 & k\\
s016 & l\\
s017 & m\\
s018 & n\\
s019 & o\\
s020 & p\\
s021 & q\\
\bottomrule
\end{tabular}


  \end{minipage}%
  \begin{minipage}{.33\linewidth}
    \centering
      
\begin{tabular}{ll}
\toprule
Old Subject & New Subject\\
\midrule
s022 & r\\
s024 & s\\
s025 & t\\
s026 & u\\
s027 & v\\
s028 & w\\
s029 & x\\
s030 & y\\
s031 & z\\
s032 & A\\
s033 & B\\
s034 & C\\
s035 & D\\
s036 & E\\
s037 & F\\
s038 & G\\
s039 & H\\
\bottomrule
\end{tabular}


    \end{minipage} %
  \begin{minipage}{.33\linewidth}
    \centering
      
\begin{tabular}{ll}
\toprule
Old Subject & New Subject\\
\midrule
s040 & I\\
s041 & J\\
s042 & K\\
s043 & L\\
s044 & M\\
s046 & N\\
s047 & O\\
s048 & P\\
s049 & Q\\
s050 & R\\
s051 & S\\
s052 & T\\
s053 & U\\
s054 & V\\
s055 & W\\
s056 & X\\
s057 & Y\\
\bottomrule
\end{tabular}


    \end{minipage} 
\end{table}

\newpage

\subsection{\texorpdfstring{\label{appendix:wilcoxon}}{}}\label{section-2}

\begin{longtable}[]{@{}lrrrr@{}}
\caption{Table of p-values for differences in SessionIndex for each
variable}\tabularnewline
\toprule
Variable & Test Statistic & P-Value & Bonferroni &
Benjamini-Hochberg\tabularnewline
\midrule
\endfirsthead
\toprule
Variable & Test Statistic & P-Value & Bonferroni &
Benjamini-Hochberg\tabularnewline
\midrule
\endhead
UD.t.i & -4.7237521 & 0.0000012 & 0.0000359 & 0.0000359\tabularnewline
DD.Shift.r.o & 2.7433474 & 0.0030408 & 0.0942653 &
0.0471326\tabularnewline
H.period & -2.7108818 & 0.0033552 & 0.1040120 & 0.0346707\tabularnewline
DD.t.i & -2.7108818 & 0.0033552 & 0.1040120 & 0.0260030\tabularnewline
H.l & -2.7108818 & 0.0033552 & 0.1040120 & 0.0208024\tabularnewline
UD.Shift.r.o & 2.2888283 & 0.0110447 & 0.3423846 &
0.0570641\tabularnewline
H.five & -1.8018436 & 0.0357850 & 1.0000000 & 0.1584765\tabularnewline
H.Shift.r & -1.8018436 & 0.0357850 & 1.0000000 &
0.1386669\tabularnewline
DD.o.a & -1.8018436 & 0.0357850 & 1.0000000 & 0.1232595\tabularnewline
UD.l.Return & -1.8018436 & 0.0357850 & 1.0000000 &
0.1109335\tabularnewline
H.t & 1.7693779 & 0.0384154 & 1.0000000 & 0.1082616\tabularnewline
UD.period.t & 1.1849962 & 0.1180095 & 1.0000000 &
0.3048578\tabularnewline
DD.i.e & 1.1849962 & 0.1180095 & 1.0000000 & 0.2814072\tabularnewline
UD.five.Shift.r & 1.1849962 & 0.1180095 & 1.0000000 &
0.2613067\tabularnewline
DD.a.n & 1.1849962 & 0.1180095 & 1.0000000 & 0.2438862\tabularnewline
UD.a.n & 1.1849962 & 0.1180095 & 1.0000000 & 0.2286433\tabularnewline
H.i & -0.9577367 & 0.1690978 & 1.0000000 & 0.3083548\tabularnewline
H.n & -0.9577367 & 0.1690978 & 1.0000000 & 0.2912240\tabularnewline
UD.n.l & -0.9577367 & 0.1690978 & 1.0000000 & 0.2758964\tabularnewline
H.Return & -0.9577367 & 0.1690978 & 1.0000000 & 0.2621016\tabularnewline
DD.period.t & 0.5356832 & 0.2960887 & 1.0000000 &
0.4370834\tabularnewline
UD.i.e & 0.5356832 & 0.2960887 & 1.0000000 & 0.4172160\tabularnewline
DD.e.five & 0.5356832 & 0.2960887 & 1.0000000 & 0.3990761\tabularnewline
UD.e.five & 0.5356832 & 0.2960887 & 1.0000000 & 0.3824480\tabularnewline
DD.five.Shift.r & 0.5356832 & 0.2960887 & 1.0000000 &
0.3671500\tabularnewline
H.a & 0.5356832 & 0.2960887 & 1.0000000 & 0.3530289\tabularnewline
DD.n.l & 0.5356832 & 0.2960887 & 1.0000000 & 0.3399537\tabularnewline
H.e & -0.1785611 & 0.4291412 & 1.0000000 & 0.4751206\tabularnewline
H.o & -0.1785611 & 0.4291412 & 1.0000000 & 0.4587371\tabularnewline
UD.o.a & -0.1785611 & 0.4291412 & 1.0000000 & 0.4434459\tabularnewline
DD.l.Return & -0.1785611 & 0.4291412 & 1.0000000 &
0.4291412\tabularnewline
\bottomrule
\end{longtable}


\end{document}
